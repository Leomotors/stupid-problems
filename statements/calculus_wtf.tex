\documentclass[11pt,a4paper]{article}

\usepackage{res/style_th}

\begin{document}

\begin{problem}{Calculus WTF}{standard input}{standard output}{4.20 seconds}{69 megabytes}

วิชาแคลคูลัสถือเป็นศาสตร์หนึ่งที่มีความสำคัญต่อโลกใบนี้มาก ผมก็คิดเช่นนั้น

ดังนั้น งานของคุณซึ่งไม่ยุ่งยากมาก คือการคำนวณหาค่าของ $\int_{0}^{\infty}\frac{1}{f(x)}dx$

โดยเพื่อความง่าย $f(x)$ เป็นพหุนามดีกรี $n$ โดยมีลักษณะดังนี้

$1)$ $f(x) > 0$ $\forall x \ge 0$

$2)$ $\int_{0}^{\infty}\frac{1}{f(x)}dx$ หาค่าได้

\InputFile

\textbf{บรรทัดแรก} ประกอบด้วยจำนวนเต็ม $T$  ($1 \leq T \leq 11$) แทนจำนวนพจน์ของ $f(x)$ \\ 
\textbf{อีก $T$ บรรทัด} ประกอบด้วยจำนวนเต็ม $2$ จำนวน ได้แก่ $c_i$ และ $d_i$, โดย $c_i$ ($c_i \ge 1$) แทนสัมประสิทธ์ของพจน์ที่มีดีกรี $d_i$ และ $d_i \ne d_j$ $\forall i\forall j$ ที่ $i\ne j$ 

\OutputFile
มี $1$ จำนวน

เป็นคำตอบของสิ่งที่ถามไป คำตอบของคุณจะถือว่าถูกต้องเมื่อ $\lvert X-A\rvert \le 10^{-4}$ โดย $A$ คือคำตอบที่คุณตอบไป และ $X$ คือคำตอบที่ถูกต้อง(มั้ง)

\Scoring
$T \le 11$, $n \le 10$, $c_i \le 10$

มีชุดทดสอบ $10$ ชุด ชุดละ $10$ คะแนน

$10$ คะแนน: $f(x) = ax^2 + c$

$10$ คะแนน: $f(x) = ax^2 + bx + c$

$30$ คะแนน: $T\le 4, n\le 3$

$50$ คะแนน: ไม่มีเงื่อนไขเพิ่มเติม

\Examples

\begin{example}
\exmp{
2
1 0
1 2
}{
1.5708
}\exmp{
3
10 0
5 1
3 2
}{
0.225061
}%
\end{example}

อธิบายตัวอย่างที่ $1$ $\int_{0}^{\infty}\frac{1}{x^2+1}dx = \frac{\pi}{2} \approx 1.5708$ โดยคำตอบที่ตอบไปคือ $1.5708$ ซึ่งอยู่ภายในความคลาดเคลื่อนของ $10^{-4}$ จึงถือว่าถูกต้อง 

อธิบายตัวอย่างที่ $2$ $\int_{0}^{\infty}\frac{1}{3x^2+5x+10}dx = \frac{\pi - 2 tan^{-1}(\sqrt{\frac{5}{19}})}{\sqrt{95}} \approx 0.225061$ โดยคำตอบที่ตอบไปคือ $0.225061$ ซึ่งอยู่ภายในความคลาดเคลื่อนของ $10^{-4}$ จึงถือว่าถูกต้อง

\end{problem}

\end{document}
