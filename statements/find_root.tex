\documentclass[11pt,a4paper]{article}

\usepackage{res/style_en}

\begin{document}

\begin{problem}{Finding the Root}{standard input}{standard output}{1 seconds}{256 megabytes}

You are given a pure function from $\mathbb{R}\rightarrow\mathbb{R}$
which in C++ is:

\begin{verbatim}
double function(double);
\end{verbatim}

Your goal is to find the root of the function, a.k.a the $x$ where $f(x) = 0$

To simplify things, the function you are dealing with is \textbf{Polynomial Function},
it may have many roots but all roots are \textbf{Integer} and you are only
required to find one of them.

Implement the following function:

\begin{verbatim}
int find_root(std::function<double(double)> f);
\end{verbatim}

This function accepts function $f$ as an argument and should returns $x$
such that $f(x) = 0$

However, the function $f$ take notes of how many times you called them and
your score depends on how often you call the function.

\Constraints

\begin{itemize}
\item All roots are in the interval $[-10^9, 10^9]$ and degree of Polynomials \textbf{do not exceed} $10$
\item $\mathcal{O} = 10$
\item Your function will be called \textbf{multiple times}, this will not exceed $100$ times.
Your score for that test case is the \textbf{average score} of all function calls.
\end{itemize}

\Scoring

\begin{center}

Let $Q$ be the number of times you have called the function $f$.

\begin{tabular}{ | c | c | }
\hline
\textbf{Condition} & \textbf{Ratio of your score to the full score of the test case} \\
\hline
Answer is \textbf{Correct} and $Q \le \mathcal{O}$ & $1$ \\
\hline
Answer is \textbf{Correct} and $Q > \mathcal{O}$ & $\frac{2*\mathcal{O}}{\mathcal{O}+Q}$ \\
\hline
Answer is \textbf{Incorrect} & $0$ \\
\hline
\end{tabular}
\end{center}

Rounding will occur at test case level, if required, it will be \textbf{rounded down}.

\textbf{Note}: If the process get killed no matter by time or memory limit exceeded,
or any runtime error, the score for that test case \textbf{will be $0$} no matter
how many times have you answered the questions correct. You may need to plan
\textit{ejection strategy} for your function.

\Subtasks

1. (1 Point) The answer of all tests are $69$

2. (5 Points) The polynomial is linear

3. (9 Points) The polynomial is parabola

4. (17 Points) All roots are in the Interval $[-1$ $000, 1$ $000]$

5. (10 Points) $\mathcal{O} = \infty$

6. (13 Points) $\mathcal{O} \in \{69$ $420, 177$ $013\}$

7. (14 Points) $\mathcal{O} = 1000$

8. (31 Points) No Additional Constraints

\Ex

Definition of polynomial function $f$: \textit{(Note: This does not reflect the real grader code.)}

\begin{verbatim}
auto f = [](double x){return x*x - 3*x + 2;};
\end{verbatim}

Then, pass the function as argument to your function.

\begin{verbatim}
find_root(f);
\end{verbatim}

Then, in your function, you call

\begin{verbatim}
f(0); // returns 2
f(1); // returns 0, is root
return 1; // correct answer
\end{verbatim}

Your function should returns $1$ or $2$ which will make $f(x)=0$

\SampleGrader

\textbf{First Line}: $T, F$ represents the number of tests and Full Score of the test case

\textbf{For each test $t_i$} has 2 lines

\begin{itemize}
\item \textbf{First Line of $t_i$}: $R$, $O$ represents number of roots and $O$
\item \textbf{Second Line of $t_i$}: $x_1, x_2, \cdots, x_R$ represents roots
\end{itemize}

The sample grader will print out the score you recieved.

\end{problem}

\end{document}
