\documentclass[11pt,a4paper]{article}

\usepackage{res/style_th}

\begin{document}

\begin{problem}{ยกเลิก 1112}{standard input}{standard output}{1 seconds}{112 megabytes}

ณ ประเทศกะลาแลนด์แห่งหนึ่ง ได้มีกฎหมายมาตราหนึ่งที่ได้ชื่อว่าเป็นมาตราซึ่งขัดกับหลักสิทธิมนุษย์ชน คนรุ่นใหม่จำนวนมากที่ได้รับการเบิกเนตร จากการมาของจักรวาลนฤมิต มีความต้องการที่จะเรียกร้องให้ยกเลิกมาตรานี้ โดยแกนนำคณะยกเลิก 1112 ได้สำรวจมาแล้วว่าในประเทศแห่งนี้ มีเมืองทั้งหมด $N$ เมือง (แทนด้วยตัวเลขกำกับเมือง $1,2,\cdots,n$) โดยในแต่ละเมืองมีคนที่อยากยกเลิกมาตรานี้ $P_i$ คน และมีถนนเชื่อมระหว่างเมือง $M$ เส้น  ซึ่งแต่ละเส้นอาจมีระยะทางที่แตกต่างกัน โดยถนนระหว่างสองเมืองใดๆ ที่ติดกัน จะมีแค่หนึ่งเส้นเท่านั้น

การรวมคะแนนเสียงของคนที่ต้องการยกเลิกมาตรานี้เป็นไปด้วยดี มีคนให้ความสนใจเป็นจำนวนมาก แต่ด้วยความที่ประเทศนี้ค่อนข้างล้าหลัง การลงชื่อจำเป็นต้องทำแบบออนไซต์เท่านั้น ดังนั้นคณะยกเลิก 1112 จำเป็นต้องเลือกเมืองมาหนึ่งเมือง เพื่อตั้งศูนย์ไว้สำหรับให้ทุกคนเดินทางมาลงชื่อ

เพื่อเป็นการอำนวยความสะดวกแก่คนที่มาลงชื่อ ทางคณะยกเลิก 1112 จำเป็นต้องเลือกเมืองที่ทำให้ ผลบวกระยะทางการเดินทางของทุกคนมีค่าน้อยที่สุด ในกรณีที่มีเมืองมากกว่าหนึ่งเมืองที่ให้คำตอบเท่ากัน ให้เลือกเมืองที่มีตัวเลขกำกับน้อยที่สุด

เนื่องจากหัวหน้าคณะยกเลิก 1112 ได้เห็นคุณซึ่งได้รับเหรียญจากการแข่งขันโอลิมปิกระดับชาติ ซึ่งค่อนข้างเชื่อมันและมอบหมายให้คุณในฐานะคนรุ่นใหม่ที่ต้องการร่วมแรงเพื่อการเปลี่ยนแปลง ให้ช่วยเขียนโปรแกรมที่มี ป ร ะ สิ ท ธิ ภ า พ เพื่อมาตอบคำถามดังกล่าวหน่อย

\InputFile

\textbf{บรรทัดแรก} ประกอบด้วยจำนวนเต็ม $N$ และ $M$  ($2 \leq N \leq 500$, $1 \leq M \leq 10,000$) \\ 
\textbf{บรรทัดที่สอง} ประกอบด้วยจำนวนเต็มบวก $N$ ตัว โดยตัวที่ $i$ แทนค่า $P_i$ ที่จำนวนคนที่ต้องการลงชื่อที่เมืองที่ $i$ มีค่าไม่เกิน $10^6$ \\ 
\textbf{อีก $M$ บรรทัด} ประกอบด้วยจำนวนเต็ม $3$ ตัว $u, v, l$ แสดงถนนระหว่างเมือง $u$ และ $v$ โดยมีระยะทาง $l$ ($l \ge 1$) 

\OutputFile
มี $2$ จำนวน

จำนวนแรกคือ เลขกำกับของเมืองที่เหมาะสมที่สุดสำหรับการตั้งศูนย์ลงชื่อยกเลิกมาตรา 1112 ตามเงื่อนไขข้างบน

จำนวนที่สองคือ ผลบวกระยะทางการเดินทางของทุกคนในกรณีที่ตอบไป

\Scoring
มีทั้งหมด $10$ ชุดทดสอบ ชุดทดสอบละ $10$ คะแนน

$N \le 500, M \le 10000, max(p_i), max(l) \le 10^6$

$10$ คะแนน : $N, max(p_i), max(l) \le 10, M \le 15$

$20$ คะแนน : $N \le 20, M \le 100$

$30$ คะแนน : $N \le 69, M \le 420$

$50$ คะแนน : ไม่มีเงื่อนไขเพิ่มเติม

\Examples

\begin{example}
\exmp{
5 10
2 9 4 3 5
2 5 10
5 1 8
1 2 5
4 2 8
3 5 7
2 3 5
1 4 5
3 1 10
4 3 2
4 5 3
}{
3 90
}\exmp{
7 10
8 1 1 1 6 2 4
3 4 6
6 3 7
3 2 2
2 7 4
2 6 1
4 5 5
2 4 8
1 7 5
3 1 8
7 5 2
}{
7 79
}%
\end{example}

\end{problem}

\end{document}
